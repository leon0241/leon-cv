\documentclass[10pt,A4,english]{article}	


%----------------------------------------------------------------------------------------
%	ENCODING
%----------------------------------------------------------------------------------------

% we use utf8 since we want to build from any machine
\usepackage[utf8]{inputenc}		
\usepackage[USenglish]{isodate}
\usepackage{fancyhdr}
\usepackage[numbers]{natbib}

%----------------------------------------------------------------------------------------
%	LOGIC
%----------------------------------------------------------------------------------------

% provides \isempty test
\usepackage{xstring, xifthen}
\usepackage{enumitem}
\usepackage[english]{babel}
\usepackage{blindtext}
\usepackage{pdfpages}
\usepackage{changepage}
%----------------------------------------------------------------------------------------
%	FONT BASICS
%----------------------------------------------------------------------------------------

% some tex-live fonts - choose your own

%\usepackage[defaultsans]{droidsans}
%\usepackage[default]{comfortaa}
%\usepackage{cmbright}
\usepackage[default]{raleway}
%\usepackage{fetamont}
%\usepackage[default]{gillius}
%\usepackage[light,math]{iwona}
%\usepackage[thin]{roboto} 

% set font default
\renewcommand*\familydefault{\sfdefault} 	
\usepackage[T1]{fontenc}

% more font size definitions
\usepackage{moresize}

%----------------------------------------------------------------------------------------
%	FONT AWESOME ICONS
%---------------------------------------------------------------------------------------- 

% include the fontawesome icon set
\usepackage{fontawesome}

% use to vertically center content
% credits to: http://tex.stackexchange.com/questions/7219/how-to-vertically-center-two-images-next-to-each-other
\newcommand{\vcenteredinclude}[1]{\begingroup
\setbox0=\hbox{\includegraphics{#1}}%
\parbox{\wd0}{\box0}\endgroup}
\newcommand{\tab}[1]{\hspace{.2\textwidth}\rlap{#1}}
% use to vertically center content
% credits to: http://tex.stackexchange.com/questions/7219/how-to-vertically-center-two-images-next-to-each-other
\newcommand*{\vcenteredhbox}[1]{\begingroup
\setbox0=\hbox{#1}\parbox{\wd0}{\box0}\endgroup}

% icon shortcut
\newcommand{\icon}[3] { 							
	\makebox(#2, #2){\textcolor{maincol}{\csname fa#1\endcsname}}
}	


% icon with text shortcut
\newcommand{\icontext}[4]{ 						
	\vcenteredhbox{\icon{#1}{#2}{#3}}  \hspace{2pt}  \parbox{0.9\mpwidth}{\textcolor{#4}{#3}}
}

% icon with website url
\newcommand{\iconhref}[5]{ 						
    \vcenteredhbox{\icon{#1}{#2}{#5}}  \hspace{2pt} \href{#4}{\textcolor{#5}{#3}}
}

% icon with email link
\newcommand{\iconemail}[5]{ 						
    \vcenteredhbox{\icon{#1}{#2}{#5}}  \hspace{2pt} \href{mailto:#4}{\textcolor{#5}{#3}}
}

%----------------------------------------------------------------------------------------
%	PAGE LAYOUT  DEFINITIONS
%----------------------------------------------------------------------------------------

% page outer frames (debug-only)
% \usepackage{showframe}		

% we use paracol to display breakable two columns
\usepackage{paracol}
\usepackage{tikzpagenodes}
\usetikzlibrary{calc}
\usepackage{lmodern}
\usepackage{multicol}
\usepackage{lipsum}
\usepackage{atbegshi}
% define page styles using geometry
\usepackage[a4paper]{geometry}

% remove all possible margins
\geometry{top=1cm, bottom=1cm, left=1cm, right=1cm}

\usepackage{fancyhdr}
\pagestyle{empty}

% space between header and content
% \setlength{\headheight}{0pt}

% indentation is zero
\setlength{\parindent}{0mm}

%----------------------------------------------------------------------------------------
%	TABLE /ARRAY DEFINITIONS
%---------------------------------------------------------------------------------------- 

% extended aligning of tabular cells
\usepackage{array}

% custom column right-align with fixed width
% use like p{size} but via x{size}
\newcolumntype{x}[1]{%
>{\raggedleft\hspace{0pt}}p{#1}}%


%----------------------------------------------------------------------------------------
%	GRAPHICS DEFINITIONS
%---------------------------------------------------------------------------------------- 

%for header image
\usepackage{graphicx}

% use this for floating figures
% \usepackage{wrapfig}
% \usepackage{float}
% \floatstyle{boxed} 
% \restylefloat{figure}

%for drawing graphics		
\usepackage{tikz}			
\usepackage{ragged2e}	
\usetikzlibrary{shapes, backgrounds,mindmap, trees}

%----------------------------------------------------------------------------------------
%	Color DEFINITIONS
%---------------------------------------------------------------------------------------- 
\usepackage{transparent}
\usepackage{color}

% primary color
\definecolor{maincol}{RGB}{ 64,64,64}

% accent color, secondary
% \definecolor{accentcol}{RGB}{ 250, 150, 10 }

% dark color
\definecolor{darkcol}{RGB}{ 70, 70, 70 }

% light color
\definecolor{lightcol}{RGB}{245,245,245}

\definecolor{accentcol}{RGB}{59,77,97}



% Package for links, must be the last package used
\usepackage[hidelinks]{hyperref}

% returns minipage width minus two times \fboxsep
% to keep padding included in width calculations
% can also be used for other boxes / environments
\newcommand{\mpwidth}{\linewidth-\fboxsep-\fboxsep}
	

\usepackage{customs}

%
%
%
%	DOCUMENT CONTENT
%
%
%
%============================================================================%
\begin{document}

\columnratio{0.31}
\setlength{\columnsep}{2.2em}
\setlength{\columnseprule}{4pt}
\colseprulecolor{white}


% LEBENSLAUF HIERE
\AtBeginShipoutFirst{\Header{CV}\Footer{1}}
\AtBeginShipout{\AtBeginShipoutAddToBox{\Header{CV}\Footer{2}}}

\newpage

\colseprulecolor{lightcol}
\columnratio{0.31}
\setlength{\columnsep}{2.2em}
\setlength{\columnseprule}{4pt}
\begin{paracol}{2}


\begin{leftcolumn}
%---------------------------------------------------------------------------------------
%	META IMAGE
%----------------------------------------------------------------------------------------
% \includegraphics[width=\linewidth]{resources/image.jpg}	%trimming relative to image size

%---------------------------------------------------------------------------------------
%	META SKILLS
%----------------------------------------------------------------------------------------
	\fcolorbox{white}{white}{\begin{minipage}[c][1.5cm][c]{1\mpwidth}
		\LARGE{\textbf{\textcolor{maincol}{Philip Empl}}} \\[2pt]
		\normalsize{ \textcolor{maincol} {Management Information Systems (M. Sc.)} }
\end{minipage}} \\
\icontext{CaretRight}{12}{XX.XX.XXXX in Los Angeles}{black}\\[6pt]
\icontext{CaretRight}{12}{german}{black}\\[6pt]
\icontext{CaretRight}{12}{unmarried}{black}\\[6pt]



\cvsection{Skills}

\cvskill{Software development} {6+ Jahre} {1} \\[-2pt]

\cvskill{Cyber security} {4+ yrs.} {0.64} \\[-2pt]

\cvskill{Internet business/ \newline E-commerce} {4+ yrs.} {0.64} \\[-2pt]

\cvskill{Web development} {2+ yrs.} {0.32} \\[-2pt]

\cvskill{Big data/ \newline Data science} {2+ yrs.} {0.32} \\[-2pt]

\cvskill{Distributed legder \newline technology} {2+ yrs.} {0.32} \\[-2pt]

\cvskill{Internet of things} {1+ yrs.} {0.16} \\[-2pt]

\cvskill{Cloud computing} {1+ yrs.} {0.16} \\[-2pt] \\

\cvskill{German} {L1} {1} \\[-2pt]

\cvskill{English} {C1} {0.9} \\[-2pt]

\newpage
%---------------------------------------------------------------------------------------
%	EDUCATION
%----------------------------------------------------------------------------------------
\cvsection{Education}

\cvmetaevent
{XX/XXXX - XX/XXXX}
{Management Information Systems (M.Sc.)}
{University of Lorem}
{\textit{Field 1 • Field 2} \newline Master's thesis: \glqq title of Master's thesis\grqq.}
\cvmetaevent
{XX/XXXX - XX/XXXX}
{Managment Information Systems (B.Sc.)}
{University of Lorem}
{\textit{Field 1 • Field 2} \newline Bachelor's thesis: \glqq title of Master's thesis\grqq.}

\cvmetaevent
{XX/XXXX - XX/XXXX}
{A-Level}
{High School Lorem}
{\textit{Field 1 • Field 2 • Field 3 • Field 4 • Field 5}.}


%
%\cvsection{Projekte}

%	\cvlist{
%		\item \hyperlink{https://github.com/philipempl/ether-twin}{\textbf{Ether-Twin.}}\\ Ethereum Applikation für Digital Twins.
%		\item \hyperlink{https://github.com/philipempl/Peter-Pan}{\textbf{Peter Pan.}}\\ Koch-App (t.b.a.).
%		\item \hyperlink{https://github.com/philipempl/Innovation-Tool}{\textbf{Innovation Tool.}}\\ Webcrawler für \hyperlink{https://ibi.de/}{Ibi}.
%		\item \hyperlink{https://github.com/philipempl/cozone}{\textbf{COZONE.}} \\ Soziales Netzwerk (t.b.a.).
%		\item \hyperlink{https://github.com/geritwagner/enlit}{\textbf{ENLIT.}}\\ Exploring new Literature (Bachelorarbeit).
%		\item \textbf{Crowdfunding.} \\Modul mit \hyperlink{https://senacor.com/}{Senacor} für \hyperlink{https://www.paydirekt.de/}{paydirekt}.
%		}

\cvsection{Interests}

\icontext{CaretRight}{12}{Guitarist}{black}\\[6pt]
\icontext{CaretRight}{12}{DIY}{black}\\[6pt]
\icontext{CaretRight}{12}{Fitness}{black}\\[6pt]
\icontext{CaretRight}{12}{Cooking}{black}\\[6pt]
\icontext{CaretRight}{12}{Travel}{black}\\[6pt]




\cvsection{Contact}

\icontext{MapMarker}{16}{Street name XX\\D-XXXXX Lorem}{black}\\[6pt]
\icontext{MobilePhone}{16}{+XX XXX XXXX XXXX}{black}\\[6pt]
\iconemail{Envelope}{16}{XXXX@XXXXX.XX}{XXXX@XXXXX.XX}{black}\\[6pt]
\iconhref{Home}{16}{XXX.XXX-XXXX.XX}{http://XXX-XXX.XXX.XX}{black}\\[6pt]
\iconhref{Github}{16}{github.com/username}{https://www.github.com/username}{black}\\[6pt]
\iconhref{Xing}{16}{xing.com/user\_name}{https://www.xing.com/profile/User_Name}{black}\\

	
%\cvqrcode{0.3}

\end{leftcolumn}
\begin{rightcolumn}
%---------------------------------------------------------------------------------------
%	TITLE  HEADER
%----------------------------------------------------------------------------------------


%---------------------------------------------------------------------------------------
%	PROFILE
%----------------------------------------------------------------------------------------
\cvsection{Biography}
\vspace{4pt}

\cvtext{\Blindtext[1]




}


%---------------------------------------------------------------------------------------
%	WORK EXPERIENCE
%----------------------------------------------------------------------------------------

\vspace{10pt}
\cvsection{Work experience}
\vspace{4pt}


\cvevent
{XX/XXXX - today}
	{Research assistant | PhD student}
	{Department of Lipsum (Prof. Name)\newline University of Lipsum}
	{Here you can describe your work content and go into detail about individual parts of this job. Possibly an enumeration by means of "," is useful. Otherwise, describe the essential content of this item in 1-2 sentences.}
	\vfill\null


\cvevent
{XX/XXXX - XX/XXXX}
	{Part-time research assistant}
	{Department of Lipsum (Prof. Name)\newline University of Lipsum}
	{Here you can describe your work content and go into detail about individual parts of this job. Possibly an enumeration by means of "," is useful. Otherwise, describe the essential content of this item in 1-2 sentences.}
	\vfill\null


	
\cvevent
{XX/XXXX - XX/XXXX}
	{Trainer - Course name}
	{Department of Lipsum (Prof. Name)\newline University of Lipsum}
	{Here you can describe your work content and go into detail about individual parts of this job. Possibly an enumeration by means of "," is useful. Otherwise, describe the essential content of this item in 1-2 sentences.}
	\vfill\null


	
\cvevent
    {XX/XXXX - XX/XXXX}
	{Trainer - Course name}
	{Department of Lipsum (Prof. Name)\newline University of Lipsum}
	{Here you can describe your work content and go into detail about individual parts of this job. Possibly an enumeration by means of ","  is useful. Otherwise, describe the essential content of this item in 1-2 sentences.}
	\vfill\null
	\vfill\null


   	
\cvevent
{XX/XXXX - XX/XXXX}
	{Trainer - Course name}
	{Department of Lipsum (Prof. Name)\newline University of Lipsum}
	{Here you can describe your work content and go into detail about individual parts of this job. Possibly an enumeration by means of "," is useful. Otherwise, describe the essential content of this item in 1-2 sentences.}
	\vfill\null

	
\cvevent
{XX/XXXX - XX/XXXX}
	{Part-time research assistant}
	{Department of Lipsum (Prof. Name)\newline University of Lipsum}
	{Here you can describe your work content and go into detail about individual parts of this job. Possibly an enumeration by means of "," is useful. Otherwise, describe the essential content of this item in 1-2 sentences.}
	\vfill\null



\cvevent
{XX/XXXX - XX/XXXX}
	{Internship}
	{Department \newline Company name (Location)}
	{Here you can describe your work content and go into detail about individual parts of this job. Possibly an enumeration by means of "," is useful. Otherwise, describe the essential content of this item in 1-2 sentences.}
	\vfill\null

\cvevent
{XX/XXXX - XX/XXXX}
	{Student assistant}
	{Department of Lipsum (Prof. Name)\newline University of Lipsum}
	{Here you can describe your work content and go into detail about individual parts of this job. Possibly an enumeration by means of "," is useful. Otherwise, describe the essential content of this item in 1-2 sentences.}
	\vfill\null


\cvevent
{XX/XXXX - XX/XXXX}
	{Trainer - Course name}
	{Department of Lipsum (Prof. Name)\newline University of Lipsum}
	{Here you can describe your work content and go into detail about individual parts of this job. Possibly an enumeration by means of "," is useful. Otherwise, describe the essential content of this item in 1-2 sentences.}
	\vfill\null



\cvevent
{XX/XXXX - XX/XXXX}
	{Working student}
	{Department \newline Company name (Location)}
	{Here you can describe your work content and go into detail about individual parts of this job. Possibly an enumeration by means of "," is useful. Otherwise, describe the essential content of this item in 1-2 sentences.}
	\vfill\null


\cvsection{Publications}

\begin{itemize}[leftmargin=*]
\item Author, G. \& Author, P. \&  Author G. (2020). "This is the title of the publication". In: \textit{Proceedings of the 28th Conference on Lipsum (LIPSUM)}, Lipsum, June 12-17, 2020.
\item Author, G. \& Author, P. \&  Author G. (2020). "This is the title of the publication". In: \textit{Proceedings of the 28th Conference on Lipsum (LIPSUM)}, Lipsum, June 12-17, 2020.
\item Author, G. \& Author, P. \&  Author G. (2020). "This is the title of the publication". In: \textit{Proceedings of the 28th Conference on Lipsum (LIPSUM)}, Lipsum, June 12-17, 2020.
\end{itemize}
% hofixes to create fake-space to ensure the whole height is used
\mbox{}
\vfill
\mbox{}
\vfill
\mbox{}
\vfill
\mbox{}
\vfill
\mbox{}
\vfill
\mbox{}
\vfill
\mbox{}


Lorem, \today     \hspace{1cm}   \hrulefill

\hspace*{30mm}\phantom{Lorem, \today }Philip Empl

\end{rightcolumn}
\end{paracol}


\end{document}
