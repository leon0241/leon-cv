\documentclass[../../cv-cs.tex]{subfiles}
\begin{document}

\cvlinkevent
{}
{Sound Designer - Singin' in the Rain}
{A1 for a student production of Singin' in the Rain, with 24 actors, 19 microphones, and a 17 piece band. Responsible for mic plots, organising mic swaps, patching and leading the SX get-in, as well as managing 2 assistant SDs and 5 sound assistants to ensure the show runs smoothly.}

\cvlinkevent
{\href{https://github.com/leon0241/Bedlam-Fringe-Showfile-25}{Github Link}}
{Base Showfile - Bedlam Fringe 2025}
{\begin{itemize}
    \item Designed and implemented a show file for the lighting plot used at the Bedlam Fringe 2025, a 90 seat Auditorium with a scaffolding rig.
    \item Used Vectorworks 2025 to create the lighting plot, and from that I created documentation, and an EOS show file with an Augment3d file, palettes, and a detailed magic sheet for shows to pre-program before the Fringe.
\end{itemize}}

\cvlinkevent
{\href{https://github.com/EdinburghUniversityTheatreCompany/bedlam-vwx}{Github Link}}
{Vectorworks - Bedlam Theatre}
{\begin{itemize}
    \item Designed a Vectorworks model of the Bedlam Theatre that can be exported to MVR for use in pre-visualisation software such as EOS Augment3d, or Vectorworks Vision. The model was designed using a basic floor plan and manual measurements. Created custom fixtures, classes, and layer structure to make the file accessible to technicians unfamiliar to the software and file.
\end{itemize}}

\cvlinkevent
{}
{Vectorworks - theSpace On The Mile}
{\begin{itemize}
    \item Designed a Vectorworks model of theSpace On The Mile, Space 1 from manual measurements during Fringe 2024. This was then used to help the technical get-in of the venue during Fringe 2025.
\end{itemize}}

\end{document}
